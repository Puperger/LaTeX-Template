\documentclass{article}

\usepackage[a4paper, margin=1.5cm]{geometry}
\usepackage[ngerman]{babel}
\usepackage{fynn}

\twocolumn % Remove this if you want a single column layout

\title{NotesTemplate}
\author{Fynn Krebser}
\date{August 2024}

\begin{document}
\noindent\Huge{Title}
 
\vspace{0.2cm}
\noindent\Large{Fynn Krebser\plusemail{fynn@krebser.net}}
\normalsize \section{Chapter}
$\ol{a}$
\vocab{hi}
$\catname{hi}$ das ist 

\begin{lemma}[Lemma von Zorn]
    Es sei $\mathcal{M}$ eine nichtleere induktiv geordnete Menge. Dann besitzt $\mathcal{M}$ ein maximales Element.
\end{lemma}
\begin{thm}[Satz von Fermat]
    Es gibt keine ganzzahligen Lösungen der Gleichung $x^n + y^n = z^n$ für $n > 2$.
\end{thm}
\begin{cor}[Kleiner Satz von Fermat]
    Für jede Primzahl $p$ und jede ganze Zahl $a$ gilt $a^p \equiv a \pmod{p}$.
\end{cor}
\begin{defn}[Bijektion]
    Eine Funktion $f: A \to B$ heißt \vocab{bijektiv}, wenn sie sowohl injektiv als auch surjektiv ist.
\end{defn}
\begin{proof}
    hello
\end{proof}
\begin{ex}
    Finde alle Lösungen der Gleichung $x^2 + y^2 = 1$.
\end{ex}
\begin{rem}
    Das ist ein Kommentar.
\end{rem}
\begin{qs}
    Ist die Menge der Primzahlen endlich?
\end{qs}
\begin{prob*}
    Finde $f$ wenn $\parderiv{x}f(x) = 7x^2$
\end{prob*}
$f=\frac 73x^3$ berechnen\worth{1}
\[
    \bra{a} \ket{b} \braket{a}{b} \bigbraket{\frac{a}{\frac{b}{c}}}{b}
.\] 
Sinus und Cosinus sind periodisch mit Periode $2\pi$.
\begin{center}
\begin{circuitikz}
\draw (0,0) to[ variable cute inductor ] (2,0); 
\end{circuitikz}
\end{center}
\end{document}

